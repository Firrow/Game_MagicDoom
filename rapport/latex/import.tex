\documentclass[12pt]{article}
\usepackage{times} % ou seulement l'un, ou l'autre, ou lmodern etc. / Pour pouvoir conserver une police vectorielle malgré les accents
\usepackage[french]{babel}
\usepackage[utf8]{inputenc} % pour pouvoir tapper é au lieu de \'{e}]
\usepackage[T1]{fontenc} % pour le rendu des accents et la césure pour les mots accentués
% \usepackage{mltex}
% \usepackage{lipsum}
\usepackage[a4paper, margin=20mm]{geometry}
\usepackage{graphicx} % Pour includegraphics
\usepackage{subcaption} % Pour pouvoir réaliser des subfigure
\usepackage{amssymb}
\usepackage{tabularx}
\usepackage{placeins} % Pour FloatBarrier
\usepackage{hyperref} % Pour transformer les section de tableofcontents par des liens clickables
\usepackage{enumitem}% pour personnaliser les puces des listes
\usepackage{subcaption}
\usepackage[acronym]{glossaries} % pour faire les glossaires et les acronymes
\usepackage{dirtree} % pour créer des arbres de systèmes de fichiers
\usepackage{rotating} % pour utiliser \begin{sidewaysfigure} qui permet d'afficher une image en paysage
%\usepackage[style=alphabetic]{biblatex} % to add bibliography

% \usepackage{tikz} % pour la frise chronologique et surement d'autres trucs
% \usetikzlibrary{timeline} % pour la frise chronologique
% pour installer le package il faut aller sur https://github.com/cfiandra/timeline
% télécharger le fichier tikzlibrarytimeleine.code.tex
% le mettre tel quel dans le dossier C:\Users\Baptiste\AppData\Roaming\MiKTeX\tex\generic\pgf\timeline
% est ce que le sous dossier pgf est vraiment important ?
% (regarder l'endroit exact dans miktex -> settings -> Directories -> Config) puis créer /timeline
% il faut ensuite actualiser les fichiers dans miktex -> le menu Tasks -> refresh file name databases
%EN CAS D'ERREURS AVEC BIBTEX found no \citation command etc mettre custom toolchain à
%TEX %ARG %DOC && %TEX %ARG %DOC
%dans atom-latex settings

%%%%%%% Parametre des en têtes %%%%%%%%
\usepackage{fancyhdr}
\usepackage{lastpage} % pour avoir le nombre total de page : page 1/3
\pagestyle{fancy}
\fancyhf{} % to clean header/footer of default style
\rfoot{Page \thepage/\pageref{LastPage}}
\renewcommand{\headrulewidth}{0pt}
\renewcommand{\footrulewidth}{0pt}
%%%%%%%%%%%%%%%%%%%%%%%%%%%%%%%%%%%%%%%
\makenoidxglossaries % obligatoire à mettre avant newglossaryentry
%\input{latex/glossary.tex}
% on peut utiliser \gls \Gls \glspl \Glspl qui font un lien vers le glossaire et mettent le mot en majuscule/pluriel
% on peut aussi utiliser \acrlong{...} \acrshort et \acrfull qui font respectivement
% BDD
% Base De Données
% Base De Données (BDD)

% Pour ajouter des points aux listes
\setitemize[1]{label=\textbullet}
\setitemize[2]{label=\textopenbullet}
