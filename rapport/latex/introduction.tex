Le jeu \textit{MagicDoom} a été réalisé dans le cadre du cours programmation pour le jeu vidéo, encadré par M. Carlos \textsc{Crispim}. Vous jouez un sorcier qui souhaite défendre ses chaudrons contre des hordes d'ennemis. Les ennemis laissent des gemmes en mourrant, qui permettent une fois récoltées et placées dans les chaudrons de lancer des sorts. Le joueur peut alors utiliser les sorts mis à sa disposition pour attaquer les vagues d'ennemis.

\subsection{Répartition des tâches}

Nous allons détailler ci-après les tâches accomplies par chacun des membres du groupe.

Pour Manon :
\begin{itemize}
    \item animations du personnage et des monstres ;
    \item ajout des sons au jeu ;
    \item création et modification des assets du jeu ;
    \item ajout de la logique de déplacement, de collision et des chaudrons ;
    \item ajout de l'interface et des paramètres.

\end{itemize}

Pour Baptiste :
\begin{itemize}
    \item code review ;
    \item debeuggage ;
    \item ajout de la logique de vie, de dégat et des gemmes ;
    \item écriture du rapport
\end{itemize}

\subsection{Commentaires dans le code}

Nous avons essayé d'utiliser des fonctions avec des noms explicites et d'apporter un maximum de clarté au code. Cette volonté implique que nous n'avons commenté notre code seulement aux endroits qui nous paraissaient nécéssaires.